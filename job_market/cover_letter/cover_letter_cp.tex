% Options for packages loaded elsewhere
\PassOptionsToPackage{unicode}{hyperref}
\PassOptionsToPackage{hyphens}{url}
\PassOptionsToPackage{dvipsnames,svgnames,x11names}{xcolor}
%
\documentclass[
  12pt,
]{letter}

\usepackage{amsmath,amssymb}
\usepackage{iftex}
\ifPDFTeX
  \usepackage[T1]{fontenc}
  \usepackage[utf8]{inputenc}
  \usepackage{textcomp} % provide euro and other symbols
\else % if luatex or xetex
  \usepackage{unicode-math}
  \defaultfontfeatures{Scale=MatchLowercase}
  \defaultfontfeatures[\rmfamily]{Ligatures=TeX,Scale=1}
\fi
\usepackage{lmodern}
\ifPDFTeX\else  
    % xetex/luatex font selection
\fi
% Use upquote if available, for straight quotes in verbatim environments
\IfFileExists{upquote.sty}{\usepackage{upquote}}{}
\IfFileExists{microtype.sty}{% use microtype if available
  \usepackage[]{microtype}
  \UseMicrotypeSet[protrusion]{basicmath} % disable protrusion for tt fonts
}{}
\makeatletter
\@ifundefined{KOMAClassName}{% if non-KOMA class
  \IfFileExists{parskip.sty}{%
    \usepackage{parskip}
  }{% else
    \setlength{\parindent}{0pt}
    \setlength{\parskip}{6pt plus 2pt minus 1pt}}
}{% if KOMA class
  \KOMAoptions{parskip=half}}
\makeatother
\usepackage{xcolor}
\usepackage[margin=1in,bottom=1in,left=1in,right=1in]{geometry}
\setlength{\emergencystretch}{3em} % prevent overfull lines
\setcounter{secnumdepth}{-\maxdimen} % remove section numbering
% Make \paragraph and \subparagraph free-standing
\ifx\paragraph\undefined\else
  \let\oldparagraph\paragraph
  \renewcommand{\paragraph}[1]{\oldparagraph{#1}\mbox{}}
\fi
\ifx\subparagraph\undefined\else
  \let\oldsubparagraph\subparagraph
  \renewcommand{\subparagraph}[1]{\oldsubparagraph{#1}\mbox{}}
\fi


\providecommand{\tightlist}{%
  \setlength{\itemsep}{0pt}\setlength{\parskip}{0pt}}\usepackage{longtable,booktabs,array}
\usepackage{calc} % for calculating minipage widths
% Correct order of tables after \paragraph or \subparagraph
\usepackage{etoolbox}
\makeatletter
\patchcmd\longtable{\par}{\if@noskipsec\mbox{}\fi\par}{}{}
\makeatother
% Allow footnotes in longtable head/foot
\IfFileExists{footnotehyper.sty}{\usepackage{footnotehyper}}{\usepackage{footnote}}
\makesavenoteenv{longtable}
\usepackage{graphicx}
\makeatletter
\def\maxwidth{\ifdim\Gin@nat@width>\linewidth\linewidth\else\Gin@nat@width\fi}
\def\maxheight{\ifdim\Gin@nat@height>\textheight\textheight\else\Gin@nat@height\fi}
\makeatother
% Scale images if necessary, so that they will not overflow the page
% margins by default, and it is still possible to overwrite the defaults
% using explicit options in \includegraphics[width, height, ...]{}
\setkeys{Gin}{width=\maxwidth,height=\maxheight,keepaspectratio}
% Set default figure placement to htbp
\makeatletter
\def\fps@figure{htbp}
\makeatother

\makeatletter
\makeatother
\makeatletter
\makeatother
\makeatletter
\@ifpackageloaded{caption}{}{\usepackage{caption}}
\AtBeginDocument{%
\ifdefined\contentsname
  \renewcommand*\contentsname{Table of contents}
\else
  \newcommand\contentsname{Table of contents}
\fi
\ifdefined\listfigurename
  \renewcommand*\listfigurename{List of Figures}
\else
  \newcommand\listfigurename{List of Figures}
\fi
\ifdefined\listtablename
  \renewcommand*\listtablename{List of Tables}
\else
  \newcommand\listtablename{List of Tables}
\fi
\ifdefined\figurename
  \renewcommand*\figurename{Figure}
\else
  \newcommand\figurename{Figure}
\fi
\ifdefined\tablename
  \renewcommand*\tablename{Table}
\else
  \newcommand\tablename{Table}
\fi
}
\@ifpackageloaded{float}{}{\usepackage{float}}
\floatstyle{ruled}
\@ifundefined{c@chapter}{\newfloat{codelisting}{h}{lop}}{\newfloat{codelisting}{h}{lop}[chapter]}
\floatname{codelisting}{Listing}
\newcommand*\listoflistings{\listof{codelisting}{List of Listings}}
\makeatother
\makeatletter
\@ifpackageloaded{caption}{}{\usepackage{caption}}
\@ifpackageloaded{subcaption}{}{\usepackage{subcaption}}
\makeatother
\makeatletter
\@ifpackageloaded{tcolorbox}{}{\usepackage[skins,breakable]{tcolorbox}}
\makeatother
\makeatletter
\@ifundefined{shadecolor}{\definecolor{shadecolor}{rgb}{.97, .97, .97}}
\makeatother
\makeatletter
\makeatother
\makeatletter
\makeatother
\ifLuaTeX
  \usepackage{selnolig}  % disable illegal ligatures
\fi
\IfFileExists{bookmark.sty}{\usepackage{bookmark}}{\usepackage{hyperref}}
\IfFileExists{xurl.sty}{\usepackage{xurl}}{} % add URL line breaks if available
\urlstyle{same} % disable monospaced font for URLs
\hypersetup{
  colorlinks=true,
  linkcolor={blue},
  filecolor={Maroon},
  citecolor={Blue},
  urlcolor={Blue},
  pdfcreator={LaTeX via pandoc}}

\author{}
\date{November 27, 2023}

\begin{document}
\begin{letter}{}
\opening{Dear Members of the Search Committee,}
\ifdefined\Shaded\renewenvironment{Shaded}{\begin{tcolorbox}[boxrule=0pt, interior hidden, borderline west={3pt}{0pt}{shadecolor}, breakable, sharp corners, enhanced, frame hidden]}{\end{tcolorbox}}\fi

I write to express my interest in your call for an Assistant Professor
in Comparative Politics. I am a postdoctoral fellow in the Department of
Political Science at McMaster University. I specialize in comparative
politics and quantitative methods. I received my PhD from the Department
of Political Science at the University of Illinois at Urbana-Champaign
under the supervision of Jake Bowers, Matt Winters, Gisela Sin, and
Avital Livny.

Substantively, I research political economy and political behavior in
Latin America, with topics including democracy, accountability, and
representation, gender and politics, and criminal and political
violence. Methodologically, I use tools from causal inference and
computational social science to develop standards to navigate research
design tradeoffs in quantitative studies. My work is published in
outlets including \emph{World Development} and the \emph{Journal of
Experimental Political Science}. My work is also under \textbf{revise
and resubmit} at the \emph{British Journal of Political Science}.

My primary agenda stems from an original data collection effort in
Brazil, combining text analysis and machine learning to construct the
most comprehensive dataset of corruption infractions at the local level.
This dataset informs several of my ongoing research programs. The first
project focuses on the unintended electoral consequences of
investigating corruption, with emphasis on how politicians'
informational advantage allows them to undertake preemptive behavior to
ward off negative reactions from the public. For example, in a piece
under review, I show how mayors with reelection incentives decrease
public spending in reaction to being randomly selected for an audit of
their use of federal funds. This suggests an attempt to minimize the
potential irregularities that audits may uncover.

In another solo-authored working paper, I argue that politicians fear
being caught in an electoral anti-corruption wave even when there is no
evidence of their own wrong-doing. I show that corruption revelation
drives mayors in nearby municipalities to switch parties more often in
an effort to secure a better platform for reelection. When politicians
are not investigated for corruption themselves but still expect
increased scrutiny in their performance, this behavior is more
cost-effective than working to improve performance in office.

The second research program focuses on the gendered electoral
consequences of investigating corruption. In a book chapter with Kelly
Senters Piazza (US Air Force Academy), we use my corruption infractions
dataset to show how corruption revelation increases the proportion of
female candidates running for mayor, but not their chances of winning
elections. We attribute this to incumbents' incentives to counter the
rise of female politicians. In another forthcoming book chapter in a
separate volume, we discuss the challenges and opportunities of
different data sources to study gender and corruption.

This program has evolved toward the gendered evaluations of
officeholders' performance in general. In a piece in \emph{World
Development}, we discuss how the COVID-19 pandemic has the potential to
promote female political representation through increased discontent
with the performance of male-led executives and by priming a health
policy issue commonly associated with women. In work under
\textbf{revise and resubmit} at the \emph{British Journal of Political
Science} with Virginia Oliveros (Tulane), Rebecca Weitz-Shapiro (Brown),
and Matt Winters (Illinois), we use a survey experiment in Argentina to
show gendered differential reactions to policy implementation
performance.

This research agenda has also led me to produce methodological work to
improve our ability to detect hard-to-observe social and political
phenomena. I focus on how scholars can navigate research design
tradeoffs before data collection. For example, in an article accepted at
the \emph{Journal of Experimental Political Science}, I introduce new
tools to assess the validity of estimates in double list experiments.
This is a variant of the list experiment that promises more precise
results but comes with under-explored questionnaire design
complications. My expertise in this subject has led collaborations on
topics ranging from support to same-sex marriage in Argentina to
criminal governance in Uruguay.

My experience as a methodology and area studies postdoc at two separate
institutions has given me the opportunity to teach courses in
comparative politics and quantitative methods. At McMaster, I teach data
analysis for public opinion and public policy. At Tulane, I taught
introduction comparative politics and a senior seminar on the challenges
of developing democracies from an evidence-based policy perspective. My
work as the methods editorial assistant for the \emph{American Political
Science Review} also puts me in a unique position to gain exposure to
the most current methods in the field, awareness of which I can
incorporate into my teaching and mentoring.

As a graduate student at Illinois, I taught an online course on the
politics of developing countries as an independent instructor and served
as a teaching assistant for its in-person version. I also served as a
teaching assistant for quantitative research methods courses at the
undergraduate and PhD levels using a flipped classroom approach. These
experiences have prepared me to teach to a diverse student body and to
adapt to both online and in-person platforms.

I am prepared to teach courses on comparative politics, Latin American
politics, comparative political behavior, gender and politics, democracy
and representation, research design, and quantitative methods. You can
find copies of current and sample syllabi in my teaching portfolio. As a
first-generation scholar, my teaching philosophy emphasizes building
flexible learning environments for students with different backgrounds
and career goals.

I believe my expertise makes me an excellent fit at the University of
Arkansas. If you have any questions, you can contact me via email or
phone.

Sincerely,

Gustavo Diaz\\
Postdoctoral Fellow\\
Department of Political Science\\
McMaster University



\vfill
\end{letter}\end{document}
