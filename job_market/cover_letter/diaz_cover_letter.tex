% Options for packages loaded elsewhere
\PassOptionsToPackage{unicode}{hyperref}
\PassOptionsToPackage{hyphens}{url}
%
\documentclass[
  11pt,
]{article}
\usepackage{lmodern}
\usepackage{amssymb,amsmath}
\usepackage{ifxetex,ifluatex}
\ifnum 0\ifxetex 1\fi\ifluatex 1\fi=0 % if pdftex
  \usepackage[T1]{fontenc}
  \usepackage[utf8]{inputenc}
  \usepackage{textcomp} % provide euro and other symbols
\else % if luatex or xetex
  \usepackage{unicode-math}
  \defaultfontfeatures{Scale=MatchLowercase}
  \defaultfontfeatures[\rmfamily]{Ligatures=TeX,Scale=1}
  \setmainfont[]{cochineal}
  \setmonofont[]{Fira Code}
\fi
% Use upquote if available, for straight quotes in verbatim environments
\IfFileExists{upquote.sty}{\usepackage{upquote}}{}
\IfFileExists{microtype.sty}{% use microtype if available
  \usepackage[]{microtype}
  \UseMicrotypeSet[protrusion]{basicmath} % disable protrusion for tt fonts
}{}
\makeatletter
\@ifundefined{KOMAClassName}{% if non-KOMA class
  \IfFileExists{parskip.sty}{%
    \usepackage{parskip}
  }{% else
    \setlength{\parindent}{0pt}
    \setlength{\parskip}{6pt plus 2pt minus 1pt}
    }
}{% if KOMA class
  \KOMAoptions{parskip=half}}
\makeatother
\usepackage{xcolor}
\IfFileExists{xurl.sty}{\usepackage{xurl}}{} % add URL line breaks if available
\urlstyle{same} % disable monospaced font for URLs
\usepackage[margin=1in]{geometry}
\setlength{\emergencystretch}{3em} % prevent overfull lines
\providecommand{\tightlist}{%
  \setlength{\itemsep}{0pt}\setlength{\parskip}{0pt}}
\setcounter{secnumdepth}{-\maxdimen} % remove section numbering

\ifluatex
  \usepackage{selnolig}  % disable illegal ligatures
\fi

\author{Gustavo Diaz}
\date{May 3, 2023}

% Jesus, okay, everything above this comment is default Pandoc LaTeX template. -----
% ----------------------------------------------------------------------------------
% I think I had assumed beamer and LaTex were somehow different templates.


\usepackage{kantlipsum}

\usepackage{abstract}
\renewcommand{\abstractname}{}    % clear the title
\renewcommand{\absnamepos}{empty} % originally center

\renewenvironment{abstract}
 {{%
    \setlength{\leftmargin}{0mm}
    \setlength{\rightmargin}{\leftmargin}%
  }%
  \relax}
 {\endlist}

\makeatletter
\def\@maketitle{%
  \newpage
%  \null
%  \vskip 2em%
%  \begin{center}%
  \let \footnote \thanks
      {\fontsize{18}{20}\selectfont\raggedright  \setlength{\parindent}{0pt} \@title \par}
    }
%\fi
\makeatother


 



%\author{\Large \vspace{0.05in} \newline\normalsize\emph{}  }


\date{}

\usepackage{titlesec}

% 
\titleformat*{\section}{\large\bfseries}
\titleformat*{\subsection}{\normalsize\itshape} % \small\uppercase
\titleformat*{\subsubsection}{\normalsize\itshape}
\titleformat*{\paragraph}{\normalsize\itshape}
\titleformat*{\subparagraph}{\normalsize\itshape}

% add some other packages ----------

% \usepackage{multicol}
% This should regulate where figures float
% See: https://tex.stackexchange.com/questions/2275/keeping-tables-figures-close-to-where-they-are-mentioned
\usepackage[section]{placeins}



\makeatletter
\@ifpackageloaded{hyperref}{}{%
\ifxetex
  \PassOptionsToPackage{hyphens}{url}\usepackage[setpagesize=false, % page size defined by xetex
              unicode=false, % unicode breaks when used with xetex
              xetex]{hyperref}
\else
  \PassOptionsToPackage{hyphens}{url}\usepackage[draft,unicode=true]{hyperref}
\fi
}

\@ifpackageloaded{color}{
    \PassOptionsToPackage{usenames,dvipsnames}{color}
}{%
    \usepackage[usenames,dvipsnames]{color}
}
\makeatother
\hypersetup{breaklinks=true,
            bookmarks=true,
            pdfauthor={ ()},
             pdfkeywords = {},  
            pdftitle={},
            colorlinks=true,
            citecolor=blue,
            urlcolor=blue,
            linkcolor=magenta,
            pdfborder={0 0 0}}
\urlstyle{same}  % don't use monospace font for urls

% Add an option for endnotes. -----



% This will better treat References as a section when using natbib
% https://tex.stackexchange.com/questions/49962/bibliography-title-fontsize-problem-with-bibtex-and-the-natbib-package



% set default figure placement to htbp
\makeatletter
\def\fps@figure{htbp}
\makeatother



\pagenumbering{gobble}

\newtheorem{hypothesis}{Hypothesis}

\usepackage{fontawesome}

\newcommand{\blankline}{\quad\pagebreak[2]}
\usepackage{graphicx}

\begin{document}



\hfill
\begin{minipage}[t]{1\textwidth}
\raggedleft%
{\bfseries Gustavo Diaz }\\[.35ex]
\emph{\small Kenneth Taylor Hall 527\\
1280 Main Street West\\
Hamilton, ON Canada} \\[.35ex]
\faPhone \hspace{1 mm} \small{+1 217 904 0581} \\ 
\faEnvelopeO \hspace{1 mm} \small{\tt \href{mailto:diazg2@mcmaster.ca}{\nolinkurl{diazg2@mcmaster.ca}}} \\ 
\faGlobe \hspace{1 mm} \small{\href{http://gustavodiaz.org}{\tt gustavodiaz.org}}\\ 
\hspace{1cm} \\
 May 3, 2023 \\ 
\end{minipage}

% \vspace*{1em} 

\vspace*{1em}

Dear Members of the Search Committee,

\vspace*{1em}

 

% May 3, 2023
% 
% \vspace*{1em} 
% 

% % \setlength{\parindent}{16pt}
% \setlength{\parskip}{0pt}
% 
I write to express my interest in your call for an Open Rank
Professorship in Computational Social Science. I am a postdoctoral
fellow in Advanced Statistical, Causal Inference, and Computational
Methodologies in the Department of Political Science at McMaster
University. I specialize in comparative politics and quantitative
methods. I received my PhD from the Department of Political Science at
the University of Illinois at Urbana-Champaign under the supervision of
Jake Bowers, Matt Winters, Gisela Sin, and Avital Livny.

My postdoc work involves building a machine learning text analysis
pipeline to understand methodological patterns in academic citations in
the social sciences. My research uses tools from computational social
science and design-based causal inference to develop standards to
navigate research design tradeoffs in the social sciences. My
contributions are informed by my substantive work on the challenges to
accountability and representation in low to middle income countries. My
work is published or forthcoming in outlets including \emph{World
Development}, the \emph{Journal of Experimental Political Science}, and
\emph{The SAGE Handbook of Research Methods in Political Science and
International Relations}.

My primary research agenda focuses on the practices that researchers can
adopt to improve statistical precision before data collection. In a
\emph{SAGE Handbook} chapter with Christopher Grady (USAID) and Jim
Kuklinski (Illinois), we discuss the merits and challenges of
increasingly complex survey experimental designs that improve precision
at the expense of ecological validity. In a solo-authored piece
forthcoming in the \emph{Journal of Experimental Political Science}, I
introduce new tools to assess the validity of estimates in double list
experiments. This is a variant of the list experiment that promises more
precise results but comes with under-explored questionnaire design
complications. In a working paper with Jake Bowers (Illinois) and
Christopher Grady, we discuss the circumstances under which researchers
should prefer biased yet precise estimators to analyze experimental
data, including applications to block-randomization and M-estimation. In
work under review with Erin Rossiter (Notre Dame), we argue how the
gains in precision from adopting good practices in experimental designs,
such as block randomization or repeated measurement of outcomes, can be
offset by explicit or implicit sample loss.

My methods work informs a substantive research program on the challenges
to accountability and representation in the Global South. The core of
this agenda is an original data collection effort in Brazil, combining
text analysis and machine learning to construct the most comprehensive
dataset of corruption infractions at the local level. This dataset
informs several of my ongoing research projects. For example, in a
solo-authored working paper, I argue that corruption revelation drives
politicians in nearby municipalities to undertake preemptive behavior to
ward off a negative reaction from their constituents. While previous
literature suggests that partisans might react strategically to
anticipate sanctions when their political party is linked to corruption,
my work shows that this behavior is even more widespread, and that
politicians fear being caught in an electoral anti-corruption wave even
when there is no evidence of their own wrong-doing.

I also maintain a research program on the gendered electoral
consequences of investigating corruption with Kelly Senters Piazza (US
Air Force Academy). In a forthcoming book chapter, we use my corruption
infractions dataset to show how corruption revelation increases the
proportion of female candidates running for mayor, but not their chances
of winning elections. We attribute this to incumbents' incentives to
counter the rise of female politicians. In another forthcoming chapter,
we discuss the challenges and opportunities of different data sources to
study gender and corruption.

This program extends toward the gendered evaluations of officeholders'
performance in general. In a piece in \emph{World Development}, we
discuss how the COVID-19 pandemic can to promote female political
representation through increased discontent with the performance of
male-led executives and by priming a health policy issue commonly
associated with women. In work under review with Virginia Oliveros
(Tulane), Rebecca Weitz-Shapiro (Brown), and Matt Winters (Illinois), we
use a survey experiment in Argentina to show gendered differential
reactions to policy implementation.

My experience as a methodology and area studies postdoc at two separate
institutions has given me the opportunity to teach courses in
comparative politics and quantitative methods. At McMaster, I teach data
analysis for public policy and public opinion. At Tulane, I taught
introduction to comparative politics and a seminar on evidence-informed
public policy to address social and political challenges in developing
democracies. My work as the methods editorial assistant for the
\emph{American Political Science Review} also exposes me to the most
current methods in the field, awareness of which I can incorporate into
my teaching and mentoring.

In my time at Illinois, I served as a teaching assistant for statistics
courses at the undergraduate and PhD levels using a flipped classroom
approach. I served as a math camp instructor for incoming graduate
students for three consecutive years and started a collaborative project
in which graduate students introduced their peers to new methods. I also
taught an online course on the politics of developing countries. These
experiences have prepared me to teach to a diverse student body, to
adapt to both online and in-person platforms, and to teach both the
theory and application of research methods.

I am prepared to teach courses on computational social science, machine
learning, data analysis and visualization, evidence informed public
policy, and causal inference. You can find copies of current and sample
syllabi in my website. As a first-generation scholar, my teaching
philosophy emphasizes building skills for students with different
backgrounds and career goals.

I believe my expertise makes me an excellent fit at Notre Dame. If you
have any questions, you can contact me via email or phone.

Sincerely,

Gustavo Diaz\\
Postdoctoral Fellow\\
Department of Political Science\\
McMaster University

\end{document}
